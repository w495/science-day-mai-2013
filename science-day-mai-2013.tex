%& -shell-escape

%% Copyright (c) 2010-2013, Илья w-495 Никитин, tvzavr
%%
%% Разрешается повторное распространение и использование
%% как в виде исходного кода, так и в двоичной форме,
%% если таковая будет получена, с изменениями или без,
%% при соблюдении следующих условий:
%%
%% * При повторном распространении исходного кода
%% должно оставаться указанное выше уведомление
%% об авторском праве, этот список условий
%% и последующий отказ от гарантий.
%% * Ни имя w-495, ни имена друзей или консультантов
%% не могут быть использованы в качестве поддержки
%% или продвижения продуктов, основанных на этом коде
%% без предварительного письменного разрешения.
%%
%% Этот код предоставлен владельцом авторских прав
%% и/или другими сторонами <<как `она есть>>
%% без какого-либо вида гарантий, выраженных явно
%% или подразумеваемых, включая, но не ограничиваясь ими,
%% (подразумеваемые) гарантии коммерческой ценности и пригодности
%% для конкретной цели. Ни в коем случае, если не требуется
%% соответствующим законом, или не установлено в устной форме,
%% ни один владелец авторских прав и ни одно другое лицо,
%% которое может изменять и/или повторно распространять программу,
%% как было сказано выше, не несёт ответственности,
%% включая любые общие, случайные, специальные
%% или последовавшие убытки, вследствие использования
%% или невозможности использования программы
%% (включая, но не ограничиваясь потерей данных,
%% или данными, ставшими неправильными, или потерями
%% принесенными из-за вас или третьих лиц,
%% или отказом программы работать совместно
%% с другими программами), даже если такой владелец или другое
%% лицо были извещены о возможности таких убытков.
%%

%% Документ нужно собирать только в XeLaTeX:
%% $>xelatex имя-файла.tex
%% Для этого должны быть установлены пакеты XeLaTeX и XeTeX
%% в TeXLive или MikTeX или иной,
%% если она поддерживает последние обновдения CTAN.

\documentclass[utf8,10pt,compress,hyperref={xetex}]{beamer}
\usepackage{styles/init}

\newcommand{\thetitlemod}{
    \ifnum\value{page}>1
        \textcolor{zblue}
            {$\insertframenumber \text{ из }
            \inserttotalframenumber$~|~<<Прикладная математика
            и физика>>}
    \else
        %
    \fi
}

\newcommand{\theauthormod}{
    \ifnum\value{page}>1
            \textcolor{zblue}{
                12 октября 2013 года: И.\,К.~Никитин
                при поддержке \href{http://www.tvzavr.ru}{tvzavr.ru}}
    \else
        \textcolor{swhite!60!black!50}{
            12 октября 2013 г.
            III Всероссийский Фестиваль науки.
            День науки в МАИ.
            Прикладная математика и физика.
            Вычислительная математика и программирование.
        }
    \fi
}

\title[\thetitlemod]
        {III Всероссийский Фестиваль науки.
        День науки в МАИ.
        Прикладная математика и физика.
        Вычислительная математика и программирование.}

\author{\theauthormod}

\date{\today}

\begin{document}
    
\begin{frame}
    \begin{center}
        { \scriptsize \color{zlightblue} \SansRoundedC
            МОСКОВСКИЙ АВИАЦИОННЫЙ ИНСТИТУТ \\
            (национальный исследовательский университет)
        } \\
        \vspace{45pt}
        { \color{swhite} \LARGE
            {\bfseries
                \begin{center}
                    \SansRoundedC
                    Прикладная математика
                \end{center}
            }
        }
        \vspace{25pt}
        \begin{tabular}{ccc}
            {
                \begin{tabular}{c}
                    \includegraphics[width=3.5cm]{./img/logo/mai-08.eps}
                \end{tabular}
            }
            &
            {\quad}
            &
            { \scriptsize \SansRoundedC
                \begin{tabular}{rl}
                    \multicolumn{2}{l}{
                        \color{swhite!60!black!50} Никитин Илья Константинович,
                    }\\
                    \multicolumn{2}{l}{
                        \color{swhite!60!black!50} асп. каф. 806 МАИ
                    }\\
                    \quad {\tiny \color{zdarkgreen!60!black!40} twitter:}  &
                        \tiny \href{http://twitter.com/w_495}{\color{zlightgreen!60!black!50} @w\_495}\\
                    \quad {\tiny \color{zdarkgreen!60!black!40} почта:}   &
                        \tiny \href{mailto:w@w-495.ru}{\color{zdarkgreen!60!black!50} w@w-495.ru}\\
                        &
                        \tiny\href{mailto:nikitin.i@tvzavr.ru}{\color{zlightgreen!60!black!50} nikitin.i@tvzavr.ru}\\
                        &
                        \tiny\href{mailto:w495@yandex-team.ru}{\color{zlightgreen!60!black!50} w495@yandex-team.ru}
                 \end{tabular}
            } \\
        \end{tabular}
    \end{center}
\end{frame}

   %% Обложка
    \section{Кто я}

\begin{frame}{Кто я}

    %\insertsectionhead
    \begin{center}
        \begin{tabular}{ccc}
            {\small \SansRoundedLightC аспирант, преподаватель}
            & \qquad &
            {\small \SansRoundedLightC разработчик}
            \\
            \\
            \includegraphics[height=2.5cm]{./img/logo/mai.eps}
            & \quad &
            \includegraphics[height=2cm]{./img/logo/tvzavr.png}
            \\
        \end{tabular}

        \vspace{7pt}

        \begin{tabular}{c}
            {\small \SansRoundedLightC разработчик}
            \\
            \\
             \includegraphics[height=1.7cm]{./img/logo/yandex.pdf}
            \\
        \end{tabular}
    \end{center}

\end{frame}
    %% Кто я
    \section{Что ждет}

\imageemptyframew{./img/wellcome.jpg}
    %% Что ждет
    \section{Зачем}

    \subsection{Люди}

\begin{frame}{Зачем идти на прикладную математику}
    \begin{center}
            \tikzstyle{peoplef} = [
                draw=green!50!black!70,thick,
                minimum height=3cm,
                minimum width=3cm,
                top color=yellow!20,
                bottom color=green!60!black!20,
                decorate,decoration={
                    random steps,segment length=2pt,amplitude=3pt
                }
            ]
            \tikzstyle{studyf} = [
                draw=red!50!black!70,thick,
                minimum height=3cm,
                minimum width=3cm,
                top color=yellow!20,
                bottom color=red!60!black!20,
                decorate,decoration={
                    random steps,segment length=2pt,amplitude=3pt
                }
            ]
            \tikzstyle{jobf} = [
                draw=blue!50!black!70,thick,
                minimum height=3cm,
                minimum width=3cm,
                top color=yellow!20,
                bottom color=blue!60!black!20,
                decorate,decoration={
                    random steps,segment length=2pt,amplitude=3pt
                }
            ]
            \begin{tikzpicture}[
                thick,node distance=3.8cm,
                text height=2.7ex,text depth=.5ex,
                auto
            ]
                \node[peoplef] (people) {
                    \SansRoundedLightC \color{green!50!black!90}
                    \begin{tabular}{c}
                        {\Huge $ \color{green!50!black!90} \lambda $} \\
                        \\
                        интересные \\
                        люди \\
                    \end{tabular}
                };
                \node[studyf, right of=people] (study) {
                    \SansRoundedLightC  \color{red!50!black!90}
                    \begin{tabular}{c}
                         {\Huge $ \color{red!50!black!90} \omega $} \\
                         \\
                        интересное \\
                        обучение \\
                    \end{tabular}
                };
                \node[jobf, right of=study] (job) {
                    \SansRoundedLightC \color{blue!50!black!90}
                    \begin{tabular}{c}
                        {\Huge $ \color{blue!50!black!90} \rho $} \\
                        \\
                        интересная \\
                        работа \\
                    \end{tabular}
                };
            \end{tikzpicture}
            { \Huge
                \[
                    {\color{green!50!black!90} \lambda } \Rightarrow
                    {\color{red!50!black!90} \omega } \Rightarrow
                    {\color{blue!50!black!90} \rho }
                \]
            }
    \end{center}
\end{frame}

     %% Люди
    \input{src/what-for/study}      %% Обучение
    \subsection{Проекты}

\begin{frame}{Интересные проекты}
    \begin{center}
        \tikzstyle{peoplef} = [
            draw=green!50!black!70,thick,
            minimum height=2cm,
            minimum width=2cm,
            top color=yellow!20,
            bottom color=green!60!black!20,
            decorate,decoration={
                random steps,segment length=2pt,amplitude=1pt
            }
        ]
        \tikzstyle{studyf} = [
            draw=red!50!black!70,thick,
            minimum height=3cm,
            minimum width=3cm,
            top color=yellow!20,
            bottom color=red!60!black!20,
            decorate,decoration={
                random steps,segment length=2pt,amplitude=3pt
            }
        ]
        \tikzstyle{jobf} = [
            draw=blue!50!black!70,thick,
            minimum height=2cm,
            minimum width=2cm,
            top color=yellow!20,
            bottom color=blue!60!black!20,
            decorate,decoration={
                random steps,segment length=2pt,amplitude=1pt
            }
        ]
        \tikzstyle{pojectsf} = [
            rectangle, rounded corners,
            thick,
            minimum size=2cm,
            draw=orange!50!black!50,
            top color=white,
            bottom color=orange!50!black!20
        ]
        \begin{tikzpicture}[
            thick,node distance=3.8cm,
            text height=2.7ex,text depth=.5ex,
            auto
        ]
            \node[peoplef] (people) {
                \SansRoundedLightC \color{green!50!black!90}
                \scriptsize
                \begin{tabular}{c}
                    {\Huge $ \color{green!50!black!90} \lambda $} \\
                    \\
                    интересные \\
                    люди \\
                \end{tabular}
            };
            \node[studyf, right of=people] (study) {
                \SansRoundedLightC  \color{red!50!black!90}
                \begin{tabular}{c}
                     {\Huge $ \color{red!50!black!90} \omega $} \\
                     \\
                    интересное \\
                    обучение \\
                \end{tabular}
            };
            \node[jobf, right of=study] (job) {
                \SansRoundedLightC \color{blue!50!black!90}
                \scriptsize
                \begin{tabular}{c}
                    {\Huge $ \color{blue!50!black!90} \rho $} \\
                    \\
                    интересная \\
                    работа \\
                \end{tabular}
            };

            \node[pojectsf, below of=study] (pojects) {
                \SansRoundedLightC \color{orange!50!black!90}
                \begin{tabular}{c}
                     {\Huge $ \color{orange!50!black!90} \pi $} \\
                     \\
                    проекты\\
                \end{tabular}
            };
            \path[->, red!50!black!90, very thick] (study) edge (pojects);
        \end{tikzpicture}
    \end{center}
\end{frame}


   %% Проекты

    %% Зачем
    \section{Проекты}

    
\subsection{Простые}

\imageemptyframew{./img/projects/simple-1.jpg}
\imageemptyframew{./img/projects/simple-2.jpg}
\imageemptyframeh{./img/projects/simple-3.jpg}
\imageemptyframeh{./img/projects/simple-4.jpg}
\imageemptyframew{./img/projects/simple-5.jpg}
     %% Простые
    \subsection{Сложные}

\imageemptyframew{./img/projects/big-1.jpg}
\imageemptyframew{./img/projects/big-2.jpg}
        %% Сложные
    
\subsection{Яркие}

\imageemptyframew{./img/projects/bright-1.jpg}
\imageemptyframew{./img/projects/bright-2.jpg}
\imageemptyframew{./img/projects/bright-3.jpg}
\imageemptyframew{./img/projects/bright-4.jpg}
\imageemptyframew{./img/projects/bright-5.jpg}
\imageemptyframew{./img/projects/bright-6.jpg}
\imageemptyframew{./img/projects/bright-7.jpg}
     %% Красочные
    \subsection{Не яркие}

\imageemptyframew{./img/projects/boring-1.jpg}
\imageemptyframew{./img/projects/boring-2.jpg}
\imageemptyframew{./img/projects/boring-3.jpg}
\imageemptyframew{./img/projects/boring-4.jpg}
     %% Не красочные
    \subsection{Очень яркие}

\imageemptyframew{./img/projects/verybright-1.jpg}
\imageemptyframew{./img/projects/verybright-2.jpg}
\imageemptyframew{./img/projects/verybright-3.jpg}
\imageemptyframew{./img/projects/verybright-4.jpg}
\imageemptyframew{./img/projects/verybright-5.jpg}
\imageemptyframew{./img/projects/verybright-6.jpg}
 %% Очень красочные
    \subsection{А что потом}

\imageemptyframew{./img/projects/end-1.jpg}
\imageemptyframew{./img/projects/end-2.jpg}
\imageemptyframeh{./img/projects/end-3.jpg}
        %% А что потом ...
    %% Проекты
    \section{Работа}

    \input{src/job/study}       %% Учебные проекты
    \input{src/job/real}        %% Взрослые проекты
    \input{src/job/success}     %% Успешные проекты
    \input{src/job/formula}     %% Формула
         %% Работа
    \section{Будущее}

    \input{src/future/how}          %% Какое?
    \subsection{Темное}

\imageemptyframeh{./img/future-dark.jpg}
         %% Темное
    \subsection{Светлое}

\imageemptyframeh{./img/future-light.jpg}
        %% Светлое
    \subsection{Красную или синюю}

\begin{frame}{Будущее}
   \orangebox{Общее технологическое Будущее}
    {\footnotesize
        \begin{itemize}
            \item[${\color{pacificorange} \Leftarrow}$]
                высокие технологии для всех;
            \item[${\color{pacificorange} \Leftarrow}$]
                развитая экономика;
            \item[${\color{pacificorange} \Leftarrow}$]
                высокий уровень жизни.
        \end{itemize}
    }
    \vspace{12pt}
    \zgreenbox{Ваше частное Будущее}
    {\footnotesize
        \begin{itemize}
            \item[${\color{zdarkgreen} \Leftarrow}$]
                обучение в самом лучше вузе Мира,
                на самой престижной кафедре;
            \item[${\color{zdarkgreen} \Leftarrow}$]
                участие в исследованиях и проектах;
            \item[${\color{zdarkgreen} \Leftarrow}$]
                глубокие знаний и навыки программирования;
            \item[${\color{zdarkgreen} \Leftarrow}$]
                престижная высокооплачиваемая работа.
        \end{itemize}
    }
    \vspace{12pt}
    Каким будет будущее, решать Вам, {\color{zdarkgreen} ... но настанет оно очень скоро.}

\end{frame}


\imageemptyframeh{./img/red-or-blue.jpg}
  %% Красную или синюю
      %% Будущее
\end{document}


